\documentclass[10pt, a4paper]{article}
\usepackage[a4paper,outer=1.5cm,inner=1.5cm,top=1.5cm,botom=1.5cm]{geometry}

%Lorem Ipusm dolor please don't leave any in you final report ;)
\usepackage{lipsum}
\usepackage{xcolor}
\usepackage{listings}
%give us the Capital H that we all know and love
\usepackage{float}
%Cool maths printing
\usepackage{amsmath}
%PseudoCode
\usepackage{algorithm2e}
\usepackage{algpseudocode}
\usepackage{algorithmicx}

\title{Department Computer Science and Software Engineering \\ Concordia University \\ COMP 352: Data Structure and Algorithms\\Assignment 1}
\author{Mihir Piyushkumar Pujara - 40025592\hspace{1em}\\m\_pujar@encs.concordia.ca}
\date{}
\begin{document}
	\maketitle
	\section*{Written Questions}
	\subsection*{Question 1}
    \subsubsection*{a)}
    {\SetAlgoNoLine%
    \begin{algorithm}
        \DontPrintSemicolon % Some LaTeX compilers require you to use\dontprintsemicolon instead
        \KwIn{Array of integer}
        \KwOut{Modified Array}
        
        $rightStart \gets 0$
        
        \eIf{$\Call{length}{input}\%2 == 0$}{
            $rightStart \gets \Call{length}{input}/2$
        }{
            $rightStart \gets (\Call{length}{input}/2) + 1$
        }
        
        $i \gets 0$
        
        \While{$i < \Call{length}{input}/2$} {
            \If{$i+1 < \Call{length}{input}/2$}{
                $temp \gets input[i]$
                
                $input[i] \gets input[i+1]$
                
                $input[i+1] \gets temp$
            }
            
            \If{$rightStart + i + 1 < \Call{Length}{input}$}{
                $input[rightStart+i+1] \gets input[rightStart+i]+input[rightStart+i+1]$
            }
            
        }
        
        \Return{$input$}\;
        \caption{MyAlgorithm(input)}
    \end{algorithm}}%
 
  
    
    \subsubsection*{b)}
    {Time complexity of my algorithm, in terms of Big-O is $O(n)$}
    \subsubsection*{c)}
    {Space complexity of my algorithm, in terms of Big-O is $O(1)\hspace{0.2cm} (O(3))$}

    \newpage
	
	\subsection*{Question 2}
	
	\subsubsection*{a) $2n^5\hspace{0.1cm}log\hspace{0.05cm}n\hspace{0.1cm}is\hspace{0.1cm}O(n^7\hspace{0.1cm}log\hspace{0.1cm}n)$}
	Proof,\\
	Here, {$f(n)$} = {2n^5log\hspace{0.1cm}n} \\\\
	{2n^5log\hspace{0.1cm}n \leq 2n^7log\hspace{0.1cm}n}, \hspace{0.5cm} for \hspace{0.2cm} n \geq 2 \\
	{so\hspace{0.1cm}any\hspace{0.1cm}n \geq 2} \\
	{2n^5log\hspace{0.1cm}n \leq 2n^7log\hspace{0.1cm}n} \\
	\rightarrow \hspace{0.1cm} $g(n) = n^7log\hspace{0.1cm}n$ 
	   
	\newpage
	
	\subsection*{Question 3}
	
	\subsubsection*{a)}
	{$f(n)$ of this algorithm is $8n+4$ for $n \geq 2$ \\}
	{$ 8n \leq 8n$ for $ n \geq 2 $ \\}
	{$ 4 \leq 4n$ for $ n \geq 2 $ \\}
	{so, for any $n \geq 2$,\\}
	{$8n+4 \leq 12n $ \rightarrow consider \hspace{.1cm} $c=12, n_0=2$ \rightarrow $ g(n) = n$ \\ \\}
	{Consequently, this\hspace{.1cm}$f(n)$\hspace{.1cm}is\hspace{.1cm} $O(n)$}
	\\\\
	{Now,\\}
	{$8n+4 \geq 8n $ for $ n \geq 2$}\\
	{$8n \geq n $ for $ n \geq 2$}\\
	{\rightarrow consider $ c = 1, n_0 = 2 $ \rightarrow $ g(n) = n$}\\
	{Consequently, this $ f(n) $ is $ O(n) $ and \hspace{.1cm} also \hspace{.1cm} is $ \Omega(n)$}
	
	\subsubsection*{b)}
	{Here, MyAlgorithm is run with input $ A = (4,10,5,1,3) $ and $ n = 5$ \\ \\}
	{After trace run this algorithm, value of A is $ (1,4,5,3,10) $ }
	\subsubsection*{c)}
	answer
	\subsubsection*{d)}
	answer
	\subsubsection*{e)}
	{}

		
\end{document}
