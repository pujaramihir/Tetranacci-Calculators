\documentclass[10pt, a4paper]{article}
\usepackage[a4paper,outer=1.5cm,inner=1.5cm,top=1.5cm,botom=1.5cm]{geometry}

%Lorem Ipusm dolor please don't leave any in you final report ;)
\usepackage{lipsum}
\usepackage{xcolor}
\usepackage{listings}
%give us the Capital H that we all know and love
\usepackage{float}
%Cool maths printing
\usepackage{amsmath}
%PseudoCode
\usepackage{algorithm2e}
\usepackage{algpseudocode}
\usepackage{algorithmicx}
\usepackage{amssymb}

\title{Department Computer Science and Software Engineering \\ Concordia University \\ COMP 352: Data Structure and Algorithms\\Assignment 1}
\author{Mihir Piyushkumar Pujara - 40025592\hspace{1em}\\m\_pujar@encs.concordia.ca}
\date{}
\begin{document}
	\maketitle
	\section*{Written Questions}
	\subsection*{Question 1}
    \subsubsection*{a)}
    {\SetAlgoNoLine%
    \begin{algorithm}[H]
        \caption{My algorithm}\label{euclid}
        \DontPrintSemicolon % Some LaTeX compilers require you to use\dontprintsemicolon instead
        \KwIn{Array of integer}
        \KwOut{Modified Array}
        
        $rightStart \gets 0$
        
        \eIf{$\Call{length}{input}\%2 == 0$}{
            $rightStart \gets \Call{length}{input}/2$
        }{
            $rightStart \gets (\Call{length}{input}/2) + 1$
        }
        
        $i \gets 0$
        
        \While{$i < \Call{length}{input}/2$} {
            \If{$i+1 < \Call{length}{input}/2$}{
                $temp \gets input[i]$
                
                $input[i] \gets input[i+1]$
                
                $input[i+1] \gets temp$
            }
            
            \If{$rightStart + i + 1 < \Call{Length}{input}$}{
                $input[rightStart+i+1] \gets input[rightStart+i]+input[rightStart+i+1]$
            }
            
        }
        
        \Return{$input$}\;
        \caption{MyAlgorithm(input)}
    \end{algorithm}}%
 
  
    
    \subsubsection*{b)}
    {Time complexity of my algorithm, in terms of Big-O is $O(n)$}
    \subsubsection*{c)}
    {Space complexity of my algorithm, in terms of Big-O is $O(1)\hspace{0.2cm} (O(3))$}

    \newpage
	
	\subsection*{Question 2}
	
	\subsubsection*{a) $2n^5\hspace{0.1cm}log\hspace{0.05cm}n\hspace{0.1cm}is\hspace{0.1cm}O(n^7\hspace{0.1cm}log\hspace{0.1cm}n)$}
	Proof,\\
	Here, {$f(n)$} = {2n^5log\hspace{0.1cm}n} \\\\
	{2n^5log\hspace{0.1cm}n \leq 2n^7log\hspace{0.1cm}n}, \hspace{0.5cm} for \hspace{0.2cm} n \geq 2 \\
	{so\hspace{0.1cm}any\hspace{0.1cm}n \geq 2} \\
	{2n^5log\hspace{0.1cm}n \leq 2n^7log\hspace{0.1cm}n} \\
	\rightarrow \hspace{0.1cm} $g(n) = n^7log\hspace{0.1cm}n$ 
	
	\subsubsection*{b)
	$10^8n^2+5n^4+7000n^3+n\hspace{0.1cm} is\hspace{0.1cm} \Theta(n^6)$}
	{Proof,\\}
	Here, {$f(n)$} = {$10^8n^2+5n^4+7000n^3+n$} \\\\
	{$ n \leq 10^8n^2 \leq 10^8n^6 , $ \hspace{.5cm}$ for\hspace{0.1cm}n>0$\\}
	{$ n \leq 5n^4 \leq 10^8n^6, $ \hspace{.5cm}$ for\hspace{0.1cm}n>0$\\}
	{$ n \leq 7000n^3 \leq 10^8n^6, $ \hspace{.5cm}$ for\hspace{0.1cm}n>0$\\}
	{$ n \leq n \leq 10^8n^6, $ \hspace{.5cm}$ for\hspace{0.1cm}n>0$\\}
	{$4n \leq 10^8n^2+5n^4+7000n^3+n  \leq 4*10^8n^6$ \rightarrow consider \hspace{.1cm} $c=4 * 10^6, n_0=1$ \\ \\}
	{So,\hspace{0.1cm} $10^8n^2+5n^4+7000n^3+n\hspace{0.1cm} is\hspace{0.1cm} \Theta(n^6)$}
	
	\subsubsection*{c)}
	Answer
	\subsubsection*{d) $0.01n^3+0.0000001n^7\hspace{0.1cm} is\hspace{0.1cm} \Theta(n^6)$}
	{Proof,\\}
	Here, {$f(n)$} = {$0.01n^3+0.0000001n^7$} \\\\
	{$ 0.01n^3 \leq 0.01n^3 \leq n^3 , $ \hspace{.5cm}$ for\hspace{0.1cm}n>0$\\}
	{$ 0.01n^3 \leq 5n^4 \nleq n^3, $ \hspace{.5cm}$ for\hspace{0.1cm}n>0$\\}
	\\
	{So,\hspace{0.1cm} $0.01n^3+0.0000001n^7\hspace{0.1cm} is\hspace{0.1cm} not \hspace{0.1cm} \Theta(n^6)$}
	
	\subsubsection*{e) $n^2+0.0000001n^5\hspace{0.1cm} is\hspace{0.1cm} \Omega(n^3)$}
	{Proof,\\}
	Here, {$f(n)$} = {$n^2+0.0000001n^5$} \\\\
	{$ n^3 \nleq n^2  , $ \hspace{.5cm}$ for\hspace{0.1cm}n>0$\\}
	\\
	{So,\hspace{0.1cm} $n^2+0.0000001n^5\hspace{0.1cm} is\hspace{0.1cm} not \hspace{0.1cm} \Omega(n^3)$}
	
	\subsubsection*{f)}
	Answer
	
	\newpage
	
	
	
	\subsection*{Question 3}
	
	\subsubsection*{a)}
	{$f(n)$ of this algorithm is $8n^2+4$ for $n \geq 2$ \\}
	{$ 8n^2 \leq 8n^2$ for $ n \geq 2 $ \\}
	{$ 4 \leq 4n^2$ for $ n \geq 2 $ \\}
	{so, for any $n \geq 2$,\\}
	{$8n^2+4 \leq 12n^2 $ \rightarrow consider \hspace{.1cm} $c=12, n_0=2$ \rightarrow $ g(n) = n^2$ \\ \\}
	{Consequently, this\hspace{.1cm}$f(n)$\hspace{.1cm}is\hspace{.1cm} $O(n^2)$}
	\\\\
	{Now,\\}
	{$8n^2+4 \geq 8n^2 $ for $ n \geq 2$}\\
	{$8n^2 \geq n^2 $ for $ n \geq 2$}\\
	{\rightarrow consider $ c = 1, n_0 = 2 $ \rightarrow $ g(n) = n$}\\
	{Consequently, this $ f(n) $ is $ O(n^2) $ and \hspace{.1cm} also \hspace{.1cm} is $ \Omega(n^2)$}
	
	\subsubsection*{b)}
	{Here, MyAlgorithm is run with input $ A = (4,10,5,1,3) $ and $ n = 5$ \\ \\}
	{After trace run this algorithm, value of A is $ (1,3,4,5,10) $ }
	\subsubsection*{c)}
	{This algorithm is sorting the given array.}\\
	{It can be asserted by checking every next element is grater or equal of current element.}
	
	
	\subsubsection*{d)}
	{Yes, the runtime of MyAlgorithm can be imporved by using partitioning the array and then sort it. Divide and conquer algorithm is very useful for this. It divides large array into small sub arrays and sort the sub arrays recursively.}
	\subsubsection*{e)}
	{Liner recursion is used in MyAlgorithm and Yes, MyAlgorithm is tail-recursive.}

	\newpage
	\section*{Programming Questions}
	\subsection*{a)}
	\subsubsection*{First recursive version}
	{\SetAlgoNoLine%
    \begin{algorithm}
        \Function{linearTetranacci}{n}
        
            \DontPrintSemicolon % Some LaTeX compilers require you to use\dontprintsemicolon instead
            \KwIn{integer number}
            \KwOut{Array of last four value of tetranacci series of given number}
            
            
            \eIf{$ n \leq 4$}{
                \Return{$ \{0,0,1,1\} $}
            }{
                $ temp \gets linearTetranacci(n - 1)$ \\
                \Return{$ \{temp[1],temp[2],temp[3],temp[1]+temp[2]+temp[3]+temp[4]\} $}
            }
        \EndFunction
        \caption{linearTetranacci(n)}
    \end{algorithm}}%
    
    \subsubsection*{Second recursive version}
    {\SetAlgoNoLine%
    \begin{algorithm}
        \Function{binaryTetranacci}{n}
        
            \DontPrintSemicolon % Some LaTeX compilers require you to use\dontprintsemicolon instead
            \KwIn{integer number}
            \KwOut{tetranacci number of given number}
            
            
            \uIf{$ n \leq 2$}{
                \Return{$ 0 $}
            }
            \uElseIf{$ n \leq 4$}{
               
                \Return{$ 1 $}
            }
            \Else{
                \Return{$(\Call{binaryTetranacci}{n-1}*2)-\Call{binaryTetranacci}{n-4}$}
            }
        \EndFunction
        \caption{linearTetranacci(n)}
    \end{algorithm}}%
    
    \subsection*{b)}
    {Here, First function has liner complexity because to calculete time of linearTetranacci(n) is calling linearTetranacci(n-1) and so on\\}
    {$T(n<=4) = O(1)$\\}
    {$T(n) = T(n-1) + O(1)$} \\
    {If we draw the recursion tree which will have depth n and intuitively figure that this function is asymptotically $O(n)$\\}
    {So, we can say that second function has linear complexity\\\\}
    {For, Second function has exponential complexity because to calculate time of tetranacci(n) we need to calcuate time of tetranacci(n-1) and tetranacci(n-4) \\}
    {$T(n<=4) = O(1)$\\}
    {$T(n) = T(n-1) + T(n-4) + O(1)$} \\
    {If We draw the recursion tree which will have depth n and intuitively figure that this function is asymptotically $O(2^n)$\\}
    {So, we can say that second function has exponential complexity}
    
    
    \subsection*{c)}
    {No, From above two algorithms no one is using tail recursion. In first algorithm i am calling function first them perform addition operation and In second algorithm in last return statement after function call i am performing arithmatic operations like multiplication and substraction.\\\\}
    {Yes, A tail-recursive version of tetranacci calculator can be designed.}
    \subsubsection*{Tail-recursive version}
    
    
\end{document}
